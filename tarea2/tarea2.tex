%%%
 %
 % Copyright (C) 2020 Ángel Iván Gladín García
 %
 % This program is free software: you can redistribute it and/or modify
 % it under the terms of the GNU General Public License as published by
 % the Free Software Foundation, either version 3 of the License, or
 % (at your option) any later version.
 %
 % This program is distributed in the hope that it will be useful,
 % but WITHOUT ANY WARRANTY; without even the implied warranty of
 % MERCHANTABILITY or FITNESS FOR A PARTICULAR PURPOSE.  See the
 % GNU General Public License for more details.
 %
 % You should have received a copy of the GNU General Public License
 % along with this program.  If not, see <http://www.gnu.org/licenses/>.
%%%

%%%%%%%%%%%%%%%%%%%%%%%%%%%%%%%%%%%%%%%%%%%%%%%%%%%%%%%%%%%%%%%%%%%%%%%%%%%%%%%%%%%%%%%%%
\documentclass[11pt,letterpaper]{article}
\usepackage[margin=.75in]{geometry}
\usepackage[utf8]{inputenc}
\usepackage[spanish]{babel}
\decimalpoint

\usepackage{listings}
\usepackage{color}
\usepackage{graphicx}
\usepackage{enumerate}
\usepackage{enumitem}
\usepackage{float}

\usepackage{longtable}
\usepackage{hyperref}
\usepackage{commath}

\usepackage{bbm}
\usepackage{dsfont}
\usepackage{mathrsfs}
\usepackage{amsmath,amsthm,amssymb}
\usepackage{mathtools}
\usepackage{longtable}

\usepackage{tikz}
\usetikzlibrary{trees}
\usepackage{verbatim}

% Set the overall layout of the tree
\tikzstyle{level 1}=[level distance=3.5cm, sibling distance=3.5cm]
\tikzstyle{level 2}=[level distance=3.5cm, sibling distance=2cm]

% Define styles for bags and leafs
\tikzstyle{bag} = [text width=4em, text centered]
\tikzstyle{end} = [circle, minimum width=3pt,fill, inner sep=0pt]
%%%%%%%%%%%%%%%%%%%%%%%%%%%%%%%%%%%%%%%%%%%%%%%%%%%%%%%%%%%%%%%%%%%%%%%%%%%%%%%%%%%%%%%%%%%%%%%%5

\usepackage{import}

\usepackage[utf8]{inputenc}

\usepackage{listings}
\usepackage{color}

\definecolor{codegreen}{rgb}{0,0.6,0}
\definecolor{codegray}{rgb}{0.5,0.5,0.5}
\definecolor{codepurple}{rgb}{0.58,0,0.82}
\definecolor{backcolour}{rgb}{0.95,0.95,0.92}

\lstdefinestyle{mystyle}{
    backgroundcolor=\color{backcolour},   
    commentstyle=\color{codegreen},
    keywordstyle=\color{magenta},
    numberstyle=\tiny\color{codegray},
    stringstyle=\color{codepurple},
    basicstyle=\footnotesize,
    breakatwhitespace=false,         
    breaklines=true,                 
    captionpos=b,                    
    keepspaces=true,                 
    numbers=left,                    
    numbersep=5pt,                  
    showspaces=false,                
    showstringspaces=false,
    showtabs=false,                  
    tabsize=2
}

\lstset{style=mystyle}
%%%%%%%%%%%%%%%%%%%%%%%%%%%%%%%%%%%%%%%%%%%%%%%%%%%%%%%%%%%%%%%%%%%%%%%%%%%%%%%%%%%%%%%%%


%%%%%%%%%%%%%%%%%%%%%%%%%%%%%%%%%%%%%%%%%%%%%%%%%%%%%%%%%%%%%%%%%%%%%%%%%%%%%%%%%%%%%%%%%
\newcommand{\Z}{\mathbb{Z}}
\newcommand{\N}{\mathbb{N}}
\newcommand{\Q}{\mathbb{Q}}
\newcommand{\R}{\mathbb{R}}
\newcommand{\Pro}{\mathds{P}}
\newcommand{\Oh}{\mathcal{O}} %% Notacion "O"
\newcommand{\lra}{\longrightarrow}
\newcommand{\ra}{\rightarrow}
\newcommand{\ord}{\text{ord}}
\newcommand{\sol}{\textbf{\underline{Solución}: }} %% Solucion
\newcommand{\af}{\textbf{\underline{Afirmación}: }}
\newcommand{\cej}{\textbf{\underline{Contraejemplo}: }}

%%%%%%%%%%%%%%%%%%%%%%%%%%%%%%%%%%%%%%%%%%%%%%%%%%%%%%%%%%%%%%%%%%%%%%%%%%%%%%%%%%%%%%%%%

\begin{document}

%%%%%%%%%%%%%%%%%%%%%%%%%%%%%%%%%%%%%%%%%%%%%%%%%%%%%%%%%%%%%%%%%%%%%%%%%%%%%%%%%%%%%%%%%
\title{
        Universidad Nacional Autónoma de México\\
        Facultad de Ciencias\\
        Álgebra Moderna I\\
    \vspace{.5cm}
    \large
        \textbf{Tarea 2}
}
\author{
    Ángel Iván Gladín García\\
    No. cuenta: 313112470\\
    \texttt{angelgladin@ciencias.unam.mx}
}
\date{5 de Febrero 2019}
\maketitle
%%%%%%%%%%%%%%%%%%%%%%%%%%%%%%%%%%%%%%%%%%%%%%%%%%%%%%%%%%%%%%%%%%%%%%%%%%%%%%%%%%%%%%%%%

%%%%%%%%%%%%%%%%%%%%%%%%%%%%%%%%%%%%%%%%%%%%%%%%%%%%%%%%%%%%%%%%%%%%%%%%%%%%%%%%%%%%%%%%%
\newtheorem{theorem}{Teorema}
\newtheorem{example}{Ejemplo}
\newtheorem{corollary}{Corolario}
\newtheorem{lemma}{Lemma}
\newtheorem{definition}{Definicion}
\newtheorem{prop}{Proposicion}
%%%%%%%%%%%%%%%%%%%%%%%%%%%%%%%%%%%%%%%%%%%%%%%%%%%%%%%%%%%%%%%%%%%%%%%%%%%%%%%%%%%%%%%%%

%%%%%%%%%%%%%%%%%%%%%%%%%%%%%%%%%%%%%%%%%%%%%%%%%%%%%%%%%%%%%%%%%%%%%%%%%%%%%%%%%%%%%%%%%
\subsection*{Ejercicio 1.}
Sea $G$ un grupo tal que $x^2 = e$ para cualquier $x \in G$, entonces $G$ es abeliano.
\begin{proof}
    Sean $a, b \in G$, por hipótesis entonces $a^2= e$ y $b^2 = e$. Si tomamos
    $(a \ast b)^2 = (a \ast b) \ast (a \ast b) = a \ast b \ast a \ast b$ se tiene así la
    siguiente igualdad,
    \[
        a \ast b \ast a \ast b = e  \tag{1}
    \]
    Si en $(1)$ multiplicamos por $a$ a la derecha y por $b$ a la izquierda se obtiene,
    \begin{align*}
        a \ast a \ast b \ast a \ast b \ast b = a \ast e \ast b && \text{Multiplicando por $a$ y $b$ en $(1)$}\\
        a^2 \ast b \ast a \ast b^2 = a \ast e \ast b && \text{Por def. de potencia}\\
        e \ast b \ast a \ast e = a \ast e \ast b && \text{Por hipótesis}\\
        b \ast a = a \ast b && \text{Simplificando}
    \end{align*}
    Ergo $G$ es abeliano.
\end{proof}

\subsection*{Ejercicio 2.}
\begin{enumerate}[label=\alph*)]
    \item Sea $G$ un grupo abeliano finito que no contiene elementos $a \not= e$ tal que $a^2 = e$. Evalúa
    $a_1 \ast a_2 \ast \cdots \ast a_n$ donde $a_1, a_2, \cdots, a_n$ es una lista sin repeticiones, de
    todos los elementos de $G$.
    \begin{proof}\af $$\prod_{a \in G} a = e$$
        Como por hipótesis los elementos $a \in G$ que no son el neutro cumplen que $a^2 = e$ que se puede
        reescribir como
        $a = a^{-1} \iff a^{-1} \ast a^2 = a^{-1} \ast e \iff a^{-1} \ast a \ast a = a^{-1} \iff e \ast a = a^{-1}$.
        También como en $a_1, a_2, \cdots, a_n$ todos los elementos son distintos (y también su inverso es único) podemos
        descomponerlo en dos productos\footnote{Se puede ``reacomodar'' los elementos del producto por ser un grupo abeliano.}
        $(a_1 \ast a_2 \ast \ldots \ast a_n)$ y $(a_1^{-1} \ast a_2^{-1} \ast \ldots \ast a_n^{-1})$ teniendo así,
        \[
            (a_1 \ast a_2 \ast \ldots \ast a_n) = (a_1 \ast a_2 \ast \ldots \ast a_n) \ast (a_1^{-1} \ast a_2^{-1} \ast \ldots \ast a_n^{-1}) = 
            (a_1 \ast a_2 \ast \ldots \ast a_n)^2  = e
        \]
    \end{proof}

    \item Pruebe el \textit{\textbf{Teorema de Wilson}}: Si $p$ es primo, entonces
    \[
        (p-1)! \equiv -1 \mod p
    \]
    \textit{Sugerencia}: Los elementos no cero de $\Z_p$ forma un grupo multiplicativo.
    \begin{proof}
        Consideremos el grupo $G$ de elementos no cero de $\Z_p$ forma un grupo multiplicativo como
        $G = ((\Z/p\Z)^\times, \cdot)$ donde $(\Z/p\Z)^\times = \{ [1], [2], \ldots, [p-1] \}$ y $(\cdot)$
        como el producto en $\Z$, también se tiene que $G$ es abeliano por $\Z$.
        
        Como $p$ es primo, tenemos que solo tiene un solo elemento de orden 2, esto es que $a^2 = 1$ teniendo 
        así que $a = \pm 1$. Lo denotaremos como $[-1]$.

        Considerando a $(p-1)!$ como el producto de los elementos de $(\Z/p\Z)^\times$ tenemos dos posibles productos,
        \[ \prod_{g \in G} g \tag{1} \]
        \[ \prod_{g \in G,\ g^2 = 1} g \tag{2} \]
        En (1) tenemos que $\forall g \in G \ \exists! g^{-1} \in G$, lo que se hace es ir ``emparejando''
        cada elemento con su inverso. En (2) justo indica que el elemento es idempotente (su propio inverso), pero 
        por la observación previo solo tenemos un elemento en $G$.

        Por lo tanto,
        \begin{align*}
            (p-1)! &\equiv -1 \mod p\\
            \prod_{g \in G} g = \prod_{g \in G,\ g^2 = 1} g = &\equiv [-1] \mod p
        \end{align*}

    \end{proof}
\end{enumerate}

\subsection*{Ejercicio 3.}
\begin{enumerate}[label=\alph*)]
    \item Si $\alpha = (1 \ 2 \ \cdots \ r-1 \ r)$, entonces $\alpha^{-1} = (r \ r-1 \ \cdots \ 2 \ 1)$.
    \begin{proof}
        Sea $\alpha \in S_n$ su composición son su inversa es $(i)$, es decir,\\
        $\alpha \circ \alpha^{-1} = id = \alpha^{-1} \circ \alpha$.
        Ahora tomando $(1 \ 2 \ \cdots \ r-1 \ r) (r \ r-1 \ \cdots \ 2 \ 1)$, fija cada elemento entre 1 y $n$.
        La composición tiene siguiente regla de correspondencia $1 \mapsto r \mapsto 1$ y para $2 \leq j \leq n$
        $j \mapsto j-1 \mapsto j$ \textbf{fijando} a cada elemento, por tanto es $(1)$.
    \end{proof}

    \item Encuentre el inverso de
    \[
        \bigl(\begin{smallmatrix}
        1 & 2 & 3 & 4 & 5 & 6 & 7 & 8 & 9\\
        6 & 4 & 1 & 2 & 5 & 3 & 8 & 9 & 7
        \end{smallmatrix}\bigr)
    \]

    Para encontrar su inverso primero mencionaremos una proposición que nos será de utilidad,
    \begin{prop}
        Si $\gamma \in S_n$ y $\gamma = \beta_i \cdots \beta_k$, entonces
        $$ \gamma^{-1} = \beta_k^{-1} \cdots \beta_1^{-1} $$
        \begin{proof} \textbf{\underline{Por inducción} } para $k \geq 2$.
            \[
                (\beta_1 \beta_2)(\beta_2^{-1} \beta_1^{-1}) =
                    \beta_1(\beta_2\beta_2^{-1})\beta_1^{-1} =
                    \beta_1 \beta^{-1} = (1)
            \]
            De manera análoga para $(\beta_2^{-1} \beta_1^{-1}) = (\beta_1 \beta_2) = (1)$.
            
            Para el caso inductivo, dea $\delta = \beta_1 \cdots \beta_k$, de modo que
            $\beta_1 \cdots \beta_k \beta{k+1} = \delta \beta_{k+1}$. Entonces
            \begin{align*}
                (\beta_1 \cdots \beta_k \beta{k+1})^{1}
                    &= (\delta \beta_{k+1})^{-1}\\
                    &= \beta_{k+1}^{-1} \delta^{-1}\\
                    &= \beta_{k+1}^{-1} (\beta_1 \cdots \beta_k)^{-1}
                    &= \beta_{k+1}^{-1} \beta_{k}^{-1} \cdots \beta_1{-1}
            \end{align*}
        \end{proof}
    \end{prop}
    Entonces, escribiendo la permutación como $\alpha = (1 \ 6 \ 3) (2 \ 4) (7 \ 8 \ 9)$, y aplicando
    la proposición anterior se tiene que
    \[
        \alpha^{-1} = (7 \ 8 \ 9)^{-1} (2 \ 4)^{-1} (1 \ 6 \ 3)^{-1} = (9 \ 8 \ 7) (4 \ 2) (3 \ 6 \ 1) = 
            (7 \ 9 \ 8) (2 \ 4) (1 \ 3 \ 6)
    \]
    
\end{enumerate}

\subsection*{Ejercicio 4.}
\textit{\textbf{Leyes de cancelación}}. En un grupo $G$, cualquiera de las ecuaciones $a \ast b = a \ast c$
y $b \ast a = c \ast a$ implica $b = c$.
\begin{proof}
    Tomando la ecuación $a \ast b = a \ast c$, teniendo que $a \in G$ entonces existe $a^{-1}$ tal que
    $a \ast a^{-1} = e$. Multiplicando por la izquierda la ecuación anterior por $a^{-1}$ se obtiene,
    \begin{align*}
        a \ast b = a \ast c && \text{Por hipótesis}\\
        a^{-1} \ast a \ast b = a^{-1} \ast a \ast c && \text{Multiplicando por $a^{-1}$ por la izq.}\\
        (a^{-1} \ast a) \ast b = (a^{-1} \ast a) \ast c && \text{Asociando}\\
        e \ast b = e \ast c && \text{Neutro}\\
        b = c
    \end{align*}
\end{proof}

\subsection*{Ejercicio 5.}
Pruebe que los siguientes cuatro permutaciones forman un grupo $\mathbf{V}$ (que es llamado el 4-grupo de Klein):
\[
    1; \qquad (1\ 2)(3\ 4); \qquad (1\ 3)(2\ 4); \qquad (1\ 4)(2\ 3);
\]
\begin{proof}
    $\mathbf{V}$ es un grupo. Para ello usaremos la tabla de Cayley.
    
    \[
    \begin{array}{l|*{4}{l}}
                 & 1            & (1\ 2)(3\ 4) & (1\ 3)(2\ 4) & (1\ 4)(2\ 3)  \\
    \hline
    1            & 1            & (1\ 2)(3\ 4) & (1\ 3)(2\ 4) & (1\ 4)(2\ 3)  \\
    (1\ 2)(3\ 4) & (1\ 2)(3\ 4) & 1            & (1\ 4)(2\ 3) & (1\ 3)(2\ 4)  \\
    (1\ 3)(2\ 4) & (1\ 3)(2\ 4) & (1\ 4)(2\ 3)  & 1            & (1\ 2)(3\ 4) \\
    (1\ 4)(2\ 3) & (1\ 4)(2\ 3) & (1\ 3)(2\ 4) & (1\ 2)(3\ 4) & 1 \\
    \end{array} 
    \]

    Y observando que en efecto está bien porque satisface la tabla de Cayley del 4-grupo de Klein $K_4$,
    \[
        \begin{array}{l|*{4}{l}}
            & e & a & b & c \\
        \hline
        e   & e & a & b & c \\
        a   & a & e & c & b \\
        b   & b & c & e & a \\
        c   & c & b & a & e \\
        \end{array} 
        \]
\end{proof}


%%%%%%%%%%%%%%%%%%%%%%%%%%%%%%%%%%%%%%%%%%%%%%%%%%%%%%%%%%%%%%%%%%%%%%%%%%%%%%%%%%%%%%%%%


%%%%%%%%%%%%%%%%%%%%%%%%%%%%%%%%%%%%%%%%%%%%%%%%%%%%%%%%%%%%%%%%%%%%%%%%%%%%%%%%%%%%%%%%%

%%%%%%%%%%%%%%%%%%%%%%%%%%%%%%%%%%%%%%%%%%%%%%%%%%%%%%%%%%%%%%%%%%%%%%%%%%%%%%%%%%%%%%%%%

\end{document}