%%%
 %
 % Copyright (C) 2020 Ángel Iván Gladín García
 %
 % This program is free software: you can redistribute it and/or modify
 % it under the terms of the GNU General Public License as published by
 % the Free Software Foundation, either version 3 of the License, or
 % (at your option) any later version.
 %
 % This program is distributed in the hope that it will be useful,
 % but WITHOUT ANY WARRANTY; without even the implied warranty of
 % MERCHANTABILITY or FITNESS FOR A PARTICULAR PURPOSE.  See the
 % GNU General Public License for more details.
 %
 % You should have received a copy of the GNU General Public License
 % along with this program.  If not, see <http://www.gnu.org/licenses/>.
%%%

%%%%%%%%%%%%%%%%%%%%%%%%%%%%%%%%%%%%%%%%%%%%%%%%%%%%%%%%%%%%%%%%%%%%%%%%%%%%%%%%%%%%%%%%%
\documentclass[11pt,letterpaper]{article}
\usepackage[margin=.75in]{geometry}
\usepackage[utf8]{inputenc}
\usepackage[spanish]{babel}
\decimalpoint

\usepackage{listings}
\usepackage{color}
\usepackage{graphicx}
\usepackage{enumerate}
\usepackage{enumitem}
\usepackage{float}

\usepackage{longtable}
\usepackage{hyperref}
\usepackage{commath}

\usepackage{bbm}
\usepackage{dsfont}
\usepackage{mathrsfs}
\usepackage{amsmath,amsthm,amssymb}
\usepackage{mathtools}
\usepackage{longtable}

\usepackage{tikz}
\usetikzlibrary{trees}
\usepackage{verbatim}

%%%%%%%%%%%%%%%%%%%%%%%%%%%%%%%%%%%%%%%%%%%%%%%%%%%%%%%%%%%%%%%%%%%%%%%%%%%%%%%%%%%%%%%%%%%%%%%%5

\usepackage{import}

\usepackage[utf8]{inputenc}

\usepackage{listings}
\usepackage{color}
%%%%%%%%%%%%%%%%%%%%%%%%%%%%%%%%%%%%%%%%%%%%%%%%%%%%%%%%%%%%%%%%%%%%%%%%%%%%%%%%%%%%%%%%%


%%%%%%%%%%%%%%%%%%%%%%%%%%%%%%%%%%%%%%%%%%%%%%%%%%%%%%%%%%%%%%%%%%%%%%%%%%%%%%%%%%%%%%%%%
\newcommand{\Z}{\mathbb{Z}}
\newcommand{\N}{\mathbb{N}}
\newcommand{\Q}{\mathbb{Q}}
\newcommand{\R}{\mathbb{R}}
\newcommand{\Pro}{\mathds{P}}
\newcommand{\Oh}{\mathcal{O}} %% Notacion "O"
\newcommand{\lra}{\longrightarrow}
\newcommand{\ra}{\rightarrow}
\newcommand{\ord}{\text{ord}}
\newcommand{\sol}{\textbf{\underline{Solución}: }} %% Solucion
\newcommand{\af}{\textbf{\underline{Afirmación}: }}
\newcommand{\cej}{\textbf{\underline{Contraejemplo}: }}

%%%%%%%%%%%%%%%%%%%%%%%%%%%%%%%%%%%%%%%%%%%%%%%%%%%%%%%%%%%%%%%%%%%%%%%%%%%%%%%%%%%%%%%%%

\begin{document}

%%%%%%%%%%%%%%%%%%%%%%%%%%%%%%%%%%%%%%%%%%%%%%%%%%%%%%%%%%%%%%%%%%%%%%%%%%%%%%%%%%%%%%%%%
\title{
        Universidad Nacional Autónoma de México\\
        Facultad de Ciencias\\
        Álgebra Moderna I\\
    \vspace{.5cm}
    \large
        \textbf{Tarea 4}
}
\author{
    Ángel Iván Gladín García\\
    No. cuenta: 313112470\\
    \texttt{angelgladin@ciencias.unam.mx}
}
\date{12 de Marzo 2020}
\maketitle
%%%%%%%%%%%%%%%%%%%%%%%%%%%%%%%%%%%%%%%%%%%%%%%%%%%%%%%%%%%%%%%%%%%%%%%%%%%%%%%%%%%%%%%%%

%%%%%%%%%%%%%%%%%%%%%%%%%%%%%%%%%%%%%%%%%%%%%%%%%%%%%%%%%%%%%%%%%%%%%%%%%%%%%%%%%%%%%%%%%
\newtheorem{theorem}{Teorema}
\newtheorem{example}{Ejemplo}
\newtheorem{corollary}{Corolario}
\newtheorem{lemma}{Lemma}
\newtheorem{definition}{Definicion}
\newtheorem{prop}{Proposicion}
%%%%%%%%%%%%%%%%%%%%%%%%%%%%%%%%%%%%%%%%%%%%%%%%%%%%%%%%%%%%%%%%%%%%%%%%%%%%%%%%%%%%%%%%%

%%%%%%%%%%%%%%%%%%%%%%%%%%%%%%%%%%%%%%%%%%%%%%%%%%%%%%%%%%%%%%%%%%%%%%%%%%%%%%%%%%%%%%%%%
\subsection*{Ejercicio 1.}
Sea $G$ un grupo y $H$ un subgrupo de $G$ de índice igual a 2. Demuestre que $H$ es un subgrupo
normal de $G$.
\begin{proof} Por contradicción.
    Es suficiente con probar que que si $h \in H$, entonces el conjugado $ghg^{-1} \in H$ para cada
    $g \in G$ (esto es para mostrar que $H \lhd G$).

    Como $H$ tiene ínidice 2, hay exactamente dos clases laterales, las cuales son $H$ y $aH$, donde
    $a \notin H$. Ahora, o bien $g \in G$ ó $g \in aH$.

    Si $g \in H$, entonces $ghg^{-1} \in H$ porque $H$ es un subgrupo.

    En el segundo caso, escribimos a $g = ax$, donde $x \in H$. Entonces\\
    $ghg^{-1} = (ax)h(ax)^{-1} = a(xhx^{-1})a^{-1} = ah'a^{-1}$, donde $h' = xhx^{-1} \in H$ 
    (donde $h'$ es un producto de tres elementos en $H$). Si $ghg^{-1} \notin H$, entonces
    $ghg^{-1} = ah'a^{-1} \in aH$, que es $ah'a^{-1} = ay$ para alguna $y \in H$. Cancelando
    $a$, tenemos $h'a^{-1} = y$, que da la contradicción $a = y^{-1}h' \in H$.

    Por tanto, si $h \in H$, cada conjugado de $h$ también vive en $H$, que es que $H$ se un
    subgrupo normal de $G$.
\end{proof}

Muestre mediante un ejemplo que si el índice de $H$ es mayor que dos, entonces $H$ no necesariamente
es normal en $G$.

Sea $G = S_3$ y $H = \{ 1, (1 \ 2) \}$ un subgrupo de índice 3. Entonces
$(1 \ 2 \ 3)H = \{ (1 \ 2 \ 3), (1 \ 3) \}$ y $H(1 \ 2 \ 3) = \{ (1 \ 2 \ 3), (1 \ 2) \}$,
por tanto $(1 \ 2 \ 3) H \not= H(1 \ 2 \ 3)$ y $H$ no es normal.


\subsection*{Ejercicio 2.}
Sea $G$ un grupo y $H$ un subgrupo de $G$ de índice igual a 2. Muestre que
$a^2 \in H$ para todo $a \in G$.

\begin{proof} Por contradicción, suponer que $a^2 \notin G$.
    
    Como $H$ tiene ínidice 2, entonces hay exactamente dos clases laterales, las cuales son $H$ y
    $aH$, donde $a \notin H$. Por tanto $G$ es la unión disjunta $G = H \sqcup aH$. Tomemos $g \in G$
    con $g \notin H$ de forma que $g = ah$ para alguna $h \in H$.

    Si $g^2 \notin H$, entonces $g^2 = ah'$, donde $h' \in H$. Teniendo así que,

    \[ g = g^{-1}g^2 = (ah)^{-1}ah' = h^{-1}a^{-1}ah' = h^{-1}h' \in H \]

    lo cual es una contradicción por suponer que $g^2 \notin G$.
\end{proof}

\subsection*{Ejercicio 3.}
Muestre que si $H \leq G$, entonces $H$ es normal en $G$ si y sólo si para todo
$x, y \in G$, $xy \in H$ si y sólo si $yx \in H$.
\begin{proof}
\hfill
\begin{itemize}
    \item[$\Longleftarrow)$] Asumimos que $H$ es un subgrupo normal de G.

    Entonces por definición de grupo normal tenemos que $h \in H$ y $g \in G$ entonces
    $ghg^{-1} \in H$.
    
    Tomemos $x, y \in G$ y $xy \in H$. Como $H$ es un subgrupo normal y $y \in G$, se sigue que,
    \[ y(xy)y^{-1} = yx(yy^{-1}) = yx \in H \]

    De manera análoga, $yx \in H$ y $x \in G$, se sigue que,
    \[ x(yx)x^{-1} = xy(xx^{-1}) = xy \in H \]

    \item[$\Longrightarrow)$] Asuminmos que para todo $x, y \in G$, $xy \in H$ si
    y sólo si $yx \in G$.

    Sea $a = yx \in H$, entonces,
    \begin{align*}
        a = yx\\
        xa = x(yx)\\
        (xa)x^{-1} = (xyx)x^{-1}\\
        xax^{-1} = xy
    \end{align*}
    Entonces vemos que $xy$ se puede escribir de la forma $xax^{-1}$ para toda $x \in G$
    y $a \in H$. Por tanto $H \lhd G$.
\end{itemize}
\end{proof}

\subsection*{Ejercicio 4.}
Demuestre que $A_n$ es un subgrupo normal de $S_n$.
\begin{proof}
    Como $[S_n : A_n] = \frac{|S_n|}{|A_n|} = \frac{n!}{n!/2} = 2$ y por el ejercicio 1, se tiene que
    si $[S_n : A_n] = 2$ entonces $A_n \lhd S_n$.
\end{proof}

%%%%%%%%%%%%%%%%%%%%%%%%%%%%%%%%%%%%%%%%%%%%%%%%%%%%%%%%%%%%%%%%%%%%%%%%%%%%%%%%%%%%%%%%%


%%%%%%%%%%%%%%%%%%%%%%%%%%%%%%%%%%%%%%%%%%%%%%%%%%%%%%%%%%%%%%%%%%%%%%%%%%%%%%%%%%%%%%%%%

%%%%%%%%%%%%%%%%%%%%%%%%%%%%%%%%%%%%%%%%%%%%%%%%%%%%%%%%%%%%%%%%%%%%%%%%%%%%%%%%%%%%%%%%%

\end{document}