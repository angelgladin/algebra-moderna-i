%%%
 %
 % Copyright (C) 2019 Ángel Iván Gladín García
 %
 % This program is free software: you can redistribute it and/or modify
 % it under the terms of the GNU General Public License as published by
 % the Free Software Foundation, either version 3 of the License, or
 % (at your option) any later version.
 %
 % This program is distributed in the hope that it will be useful,
 % but WITHOUT ANY WARRANTY; without even the implied warranty of
 % MERCHANTABILITY or FITNESS FOR A PARTICULAR PURPOSE.  See the
 % GNU General Public License for more details.
 %
 % You should have received a copy of the GNU General Public License
 % along with this program.  If not, see <http://www.gnu.org/licenses/>.
%%%

%%%%%%%%%%%%%%%%%%%%%%%%%%%%%%%%%%%%%%%%%%%%%%%%%%%%%%%%%%%%%%%%%%%%%%%%%%%%%%%%%%%%%%%%%
\documentclass[11pt,letterpaper]{article}
\usepackage[margin=.75in]{geometry}
\usepackage[utf8]{inputenc}
\usepackage[spanish]{babel}
\decimalpoint

\usepackage{listings}
\usepackage{color}
\usepackage{graphicx}
\usepackage{enumerate}
\usepackage{enumitem}
\usepackage{float}

\usepackage{longtable}
\usepackage{hyperref}
\usepackage{commath}

\usepackage{bbm}
\usepackage{dsfont}
\usepackage{mathrsfs}
\usepackage{amsmath,amsthm,amssymb}
\usepackage{mathtools}
\usepackage{longtable}

\usepackage{tikz}
\usetikzlibrary{trees}
\usepackage{verbatim}

% Set the overall layout of the tree
\tikzstyle{level 1}=[level distance=3.5cm, sibling distance=3.5cm]
\tikzstyle{level 2}=[level distance=3.5cm, sibling distance=2cm]

% Define styles for bags and leafs
\tikzstyle{bag} = [text width=4em, text centered]
\tikzstyle{end} = [circle, minimum width=3pt,fill, inner sep=0pt]
%%%%%%%%%%%%%%%%%%%%%%%%%%%%%%%%%%%%%%%%%%%%%%%%%%%%%%%%%%%%%%%%%%%%%%%%%%%%%%%%%%%%%%%%%%%%%%%%5

\usepackage{import}

\usepackage[utf8]{inputenc}

\usepackage{listings}
\usepackage{color}

\definecolor{codegreen}{rgb}{0,0.6,0}
\definecolor{codegray}{rgb}{0.5,0.5,0.5}
\definecolor{codepurple}{rgb}{0.58,0,0.82}
\definecolor{backcolour}{rgb}{0.95,0.95,0.92}

\lstdefinestyle{mystyle}{
    backgroundcolor=\color{backcolour},   
    commentstyle=\color{codegreen},
    keywordstyle=\color{magenta},
    numberstyle=\tiny\color{codegray},
    stringstyle=\color{codepurple},
    basicstyle=\footnotesize,
    breakatwhitespace=false,         
    breaklines=true,                 
    captionpos=b,                    
    keepspaces=true,                 
    numbers=left,                    
    numbersep=5pt,                  
    showspaces=false,                
    showstringspaces=false,
    showtabs=false,                  
    tabsize=2
}

\lstset{style=mystyle}
%%%%%%%%%%%%%%%%%%%%%%%%%%%%%%%%%%%%%%%%%%%%%%%%%%%%%%%%%%%%%%%%%%%%%%%%%%%%%%%%%%%%%%%%%


%%%%%%%%%%%%%%%%%%%%%%%%%%%%%%%%%%%%%%%%%%%%%%%%%%%%%%%%%%%%%%%%%%%%%%%%%%%%%%%%%%%%%%%%%
\newcommand{\Z}{\mathbb{Z}}
\newcommand{\N}{\mathbb{N}}
\newcommand{\Q}{\mathbb{Q}}
\newcommand{\R}{\mathbb{R}}
\newcommand{\Pro}{\mathds{P}}
\newcommand{\Oh}{\mathcal{O}} %% Notacion "O"
\newcommand{\lra}{\longrightarrow}
\newcommand{\ra}{\rightarrow}
\newcommand{\ord}{\text{ord}}
\newcommand{\sol}{\textbf{\underline{Solución}: }} %% Solucion
\newcommand{\af}{\textbf{\underline{Afirmación}: }}
\newcommand{\cej}{\textbf{\underline{Contraejemplo}: }}

%%%%%%%%%%%%%%%%%%%%%%%%%%%%%%%%%%%%%%%%%%%%%%%%%%%%%%%%%%%%%%%%%%%%%%%%%%%%%%%%%%%%%%%%%

\begin{document}

%%%%%%%%%%%%%%%%%%%%%%%%%%%%%%%%%%%%%%%%%%%%%%%%%%%%%%%%%%%%%%%%%%%%%%%%%%%%%%%%%%%%%%%%%
\title{
        Universidad Nacional Autónoma de México\\
        Facultad de Ciencias\\
        Álgebra Moderna I\\
    \vspace{.5cm}
    \large
        \textbf{Tarea 1}
}
\author{
    Ángel Iván Gladín García\\
    No. cuenta: 313112470\\
    \texttt{angelgladin@ciencias.unam.mx}
}
\date{5 de Febrero 2019}
\maketitle
%%%%%%%%%%%%%%%%%%%%%%%%%%%%%%%%%%%%%%%%%%%%%%%%%%%%%%%%%%%%%%%%%%%%%%%%%%%%%%%%%%%%%%%%%

%%%%%%%%%%%%%%%%%%%%%%%%%%%%%%%%%%%%%%%%%%%%%%%%%%%%%%%%%%%%%%%%%%%%%%%%%%%%%%%%%%%%%%%%%
\newtheorem{theorem}{Teorema}
\newtheorem{example}{Ejemplo}
\newtheorem{corollary}{Corolario}
\newtheorem{lemma}{Lemma}
\newtheorem{definition}{Definicion}
\newtheorem{prop}{Proposicion}
%%%%%%%%%%%%%%%%%%%%%%%%%%%%%%%%%%%%%%%%%%%%%%%%%%%%%%%%%%%%%%%%%%%%%%%%%%%%%%%%%%%%%%%%%

%%%%%%%%%%%%%%%%%%%%%%%%%%%%%%%%%%%%%%%%%%%%%%%%%%%%%%%%%%%%%%%%%%%%%%%%%%%%%%%%%%%%%%%%%
\subsection*{Ejercicio 1.}

\begin{enumerate}[label=\alph*)]
    \item Sea $\alpha = (i_0 \ i_1 \ \cdots \ i_{r-1})$ un $r$-ciclo. Para cualquier $i,k \geq 0$, pruebe que
    $\alpha^k(i_j) = i_{k+j}$.
    \begin{proof}
        Notemos que $\alpha^k$ es la composición de $\alpha$ $k$-veces. Haciendo la siguiente observación de que
        $i_{j+1} = \alpha(i_{j})$, $i_{j+2} = \alpha(i_{j+1}) = \alpha(\alpha(i_{j})) = \alpha^2(i_{j})$,
        $i_{j+3} = \alpha(i_{j+2}) = \alpha(\alpha^2(i_{j})) = \alpha^3(i_{j})$ y así para toda
        $k \leq r-1$, (cabe hacer la observación de que el subíndice de $i$, que es $j + l$ para
        algún $l$ se toma $j + l$ módulo $r$) construyendo así hasta $k$ se sigue que,
        \[
            i_{j+k} = \alpha^k(i_j)
        \]
        Como $\alpha(i_{j-1}) = i_j$, se puede ver que $i_{j+k} = \alpha^k(i_j)$ si el
        subíndice $j+k$ de $i$ es tomado módulo $r$.
    \end{proof}

    \item Pruebe que si $\alpha$ es un $r$-ciclo, entonces $\alpha^r = 1$, pero que $\alpha^k \not= 1$ para
    cualquier entero positivo $k < r$.
    \begin{proof}
        Sea $\alpha$ una permutación de la forma $\alpha = (i_0 \ i_1 \ \cdots \ i_{r-1})$, tenemos por el inciso
        anterior previamente demostrado que dada una premutación $\beta$ se tiene que $\beta^k(i_j) = i_{k+j}$.
        Entonces si tomamos un $j$ tal que $0 \leq j < r$, se sigue que $\alpha^r(i_j) = i_{r+j} = i_{j}$ (esto pasa porque
        $r+j = j$ en $\Z_{r}$), lo que significa que $\alpha$ deja fijos a los elementos. Ergo $\alpha^r = 1$.

        Para demostrar que $\alpha^k \not= 1$ para cualquier entero positivo $k < r$, tomemos un $j$ tal que $0 \leq j < r$,
        entonces $\alpha^k(i_j) = i_{j+k}$, y como por hipótesis $k < r$ entonces $i_k \not= i_{j+k}$.
        Por tanto $\alpha^k \not= 1$.
    \end{proof}

    \item Si $\alpha = \beta_1 \beta_2 \cdots \beta_m$ es un producto de $r_i$-ciclos $\beta_i$ disjuntos, entonces
    el más pequeño entero positivo $l$ con $\alpha^l = 1$ es el mínimo cómun múltiplo de
    $\{ r_1, r_2, \ldots, r_m \}$.
    \begin{proof}
        Como por hipótesis los $\beta_i$ son disjuntos, podemos usar el hecho de que conmutan para poder hacer lo siguiente,
        \[
            \alpha^n = (\beta_1 \beta_2 \cdots \beta_m)^n = \beta_1^n \beta_2^n \cdots \beta_m^n
        \]
        para toda $n \in \Z$. Supongamos que $\alpha^n = 1$ si y sólo si $\beta_i^n = 1$ para $1 \leq i \leq m$,
        Como $\alpha$ es un $n$-ciclo y $\beta_i$ es un $r_i$-ciclo, se sigue que que todo $r_i$ divide a $n$, y éso
        a su vez implica que el $m.c.m.(r_1, r_2, \ldots, r_m) \mid n$. Como $\beta_i^n = 1$ para $1 \leq i \leq m$ 
        justo aseguramos que $n$ es el entero más pequeño. Renombrando a $m.c.m.(r_1, r_2, \ldots, r_m) = l = n$,
        se sigue que $l$ es el entero mas pequeño positivo tal que $\alpha^l = 1$.
    \end{proof}
\end{enumerate}

\subsection*{Ejercicio 2.}
Sea $p$ un primo y sea $\alpha \in S_n$. Si $\alpha^p = 1$, entonces o bien $\alpha = 1$, $\alpha$
es un $p$-ciclo, o $\alpha$ es un producto de $p$-ciclos disjuntos. En particular, so $\alpha^2 = 1$,
entonces o bien $\alpha = 1$, $\alpha$ es una transposición o $\alpha$ es un producto de transposiciones
disjuntas.
\begin{proof}
    Por casos. Si $\alpha^p = 1$ entonces,
    \begin{enumerate}
        \item $\alpha = 1$.
        
        Si descomponemos a $\alpha$ como producto de dos ciclos disjuntos (esto porque como $p$ es primo)
        quedando así como $\alpha = \beta_1 \ \beta_2$ donde $\beta_1 = id$ y $\beta_2 = (1 \ 2 \ \cdots \ p) = \alpha$,
        entonces $\alpha^p = (\beta_1 \ \beta_2)^p = \beta_1^p \ \alpha^p = id \ \alpha_2^p$, por el inciso $1.2)$ podemos
        podemos expresar a la identidad como $\alpha^p$, entonces se sigue que $\alpha^p = \alpha \cdot \alpha^p = \alpha^{p+1}$
        y multiplicando por $\alpha^{-p}$ la igualdad se sigue que $\alpha^{-p} \cdot \alpha^p = \alpha^{-p} \cdot \alpha^{p+1}$
        teniendo así que $id = \alpha^p$.

        \item $\alpha$ es un $p$-ciclo.
        
        Análogo al anterior.

        \item $\alpha$ es un producto de $p$-ciclos disjuntos.
        
        Por el inciso $1.a)$ sabemos que $\alpha^p(i_k) = i_{k+p} = i_k$ lo que significa que lo deja fijos
        a todos los elementos,
        también podemos expresar $p$-ciclos disjuntos de la forma $id = (1)(2)\cdots(p)$, y ambos los dejan fijos
        a los $i's$, por lo que ambos son la tienen la misma regla de correspondecia.
    \end{enumerate}
\end{proof}

\begin{figure}[H]
    \centering
    \includegraphics[scale=0.4]{gato.jpeg}
\end{figure}

%%%%%%%%%%%%%%%%%%%%%%%%%%%%%%%%%%%%%%%%%%%%%%%%%%%%%%%%%%%%%%%%%%%%%%%%%%%%%%%%%%%%%%%%%


%%%%%%%%%%%%%%%%%%%%%%%%%%%%%%%%%%%%%%%%%%%%%%%%%%%%%%%%%%%%%%%%%%%%%%%%%%%%%%%%%%%%%%%%%

%%%%%%%%%%%%%%%%%%%%%%%%%%%%%%%%%%%%%%%%%%%%%%%%%%%%%%%%%%%%%%%%%%%%%%%%%%%%%%%%%%%%%%%%%

\end{document}