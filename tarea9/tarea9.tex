%%%
 %
 % Copyright (C) 2020 Ángel Iván Gladín García
 %
 % This program is free software: you can redistribute it and/or modify
 % it under the terms of the GNU General Public License as published by
 % the Free Software Foundation, either version 3 of the License, or
 % (at your option) any later version.
 %
 % This program is distributed in the hope that it will be useful,
 % but WITHOUT ANY WARRANTY; without even the implied warranty of
 % MERCHANTABILITY or FITNESS FOR A PARTICULAR PURPOSE.  See the
 % GNU General Public License for more details.
 %
 % You should have received a copy of the GNU General Public License
 % along with this program.  If not, see <http://www.gnu.org/licenses/>.
%%%

%%%%%%%%%%%%%%%%%%%%%%%%%%%%%%%%%%%%%%%%%%%%%%%%%%%%%%%%%%%%%%%%%%%%%%%%%%%%%%%%%%%%%%%%%
\documentclass[letterpaper]{article}
\usepackage[margin=.75in]{geometry}
\usepackage[utf8]{inputenc}
\usepackage[spanish]{babel}
\decimalpoint

\usepackage{listings}
\usepackage{color}
\usepackage{graphicx}
\usepackage{enumerate}
\usepackage{enumitem}
\usepackage{float}

\usepackage{longtable}
\usepackage{hyperref}
\usepackage{commath}

\usepackage{bbm}
\usepackage{dsfont}
\usepackage{mathrsfs}
\usepackage{amsmath,amsthm,amssymb}
\usepackage{mathtools}
\usepackage{longtable}

\usepackage{tikz}
\usetikzlibrary{trees}
\usepackage{verbatim}
\usepackage{wasysym}

%%%%%%%%%%%%%%%%%%%%%%%%%%%%%%%%%%%%%%%%%%%%%%%%%%%%%%%%%%%%%%%%%%%%%%%%%%%%%%%%%%%%%%%%%%%%%%%%5

\usepackage{import}

\usepackage[utf8]{inputenc}

\usepackage{listings}
\usepackage{color}

%%%%%%%%%%%%%%%%%%%%%%%%%%%%%%%%%%%%%%%%%%%%%%%%%%%%%%%%%%%%%%%%%%%%%%%%%%%%%%%%%%%%%%%%%


%%%%%%%%%%%%%%%%%%%%%%%%%%%%%%%%%%%%%%%%%%%%%%%%%%%%%%%%%%%%%%%%%%%%%%%%%%%%%%%%%%%%%%%%%
\newcommand{\Z}{\mathbb{Z}}
\newcommand{\N}{\mathbb{N}}
\newcommand{\Q}{\mathbb{Q}}
\newcommand{\R}{\mathbb{R}}
\newcommand{\Pro}{\mathds{P}}
\newcommand{\Oh}{\mathcal{O}} %% Notacion "O"
\newcommand{\lra}{\longrightarrow}
\newcommand{\ra}{\rightarrow}
\newcommand{\ord}{\text{ord}}
\newcommand{\sol}{\textbf{\underline{Solución}: }} %% Solucion
\newcommand{\af}{\textbf{\underline{Afirmación}: }}
\newcommand{\cej}{\textbf{\underline{Contraejemplo}: }}

%%%%%%%%%%%%%%%%%%%%%%%%%%%%%%%%%%%%%%%%%%%%%%%%%%%%%%%%%%%%%%%%%%%%%%%%%%%%%%%%%%%%%%%%%

\begin{document}

%%%%%%%%%%%%%%%%%%%%%%%%%%%%%%%%%%%%%%%%%%%%%%%%%%%%%%%%%%%%%%%%%%%%%%%%%%%%%%%%%%%%%%%%%
\title{
    \vspace{-2.2em}
        Universidad Nacional Autónoma de México\\
        Facultad de Ciencias\\
        Álgebra Moderna I\\
    \vspace{.5cm}
    \large
        \textbf{Tarea 9}
}
\author{
    Ángel Iván Gladín García\\
    No. cuenta: 313112470\\
    \texttt{angelgladin@ciencias.unam.mx}
}
\date{13 de Mayo 2020}
\maketitle
%%%%%%%%%%%%%%%%%%%%%%%%%%%%%%%%%%%%%%%%%%%%%%%%%%%%%%%%%%%%%%%%%%%%%%%%%%%%%%%%%%%%%%%%%

%%%%%%%%%%%%%%%%%%%%%%%%%%%%%%%%%%%%%%%%%%%%%%%%%%%%%%%%%%%%%%%%%%%%%%%%%%%%%%%%%%%%%%%%%
\newtheorem{theorem}{Teorema}
\newtheorem{example}{Ejemplo}
\newtheorem{corollary}{Corolario}
\newtheorem{lemma}{Lemma}
\newtheorem{definition}{Definicion}
\newtheorem{prop}{Proposicion}
%%%%%%%%%%%%%%%%%%%%%%%%%%%%%%%%%%%%%%%%%%%%%%%%%%%%%%%%%%%%%%%%%%%%%%%%%%%%%%%%%%%%%%%%%

%%%%%%%%%%%%%%%%%%%%%%%%%%%%%%%%%%%%%%%%%%%%%%%%%%%%%%%%%%%%%%%%%%%%%%%%%%%%%%%%%%%%%%%%%
\subsection*{Ejercicio 1. (25 puntos)}
Sean $X$ un $G$-conjunto, $x, y \in X$ y $g \in G$. Supongamos que $y = gx$. Demuestre que
$G_y = gG_x g^{-1}$; concluya que $|G_x| = |G_y|$.

\begin{proof}
\hfill
\begin{itemize}
    \item[$(\subseteq)$]
    Si $a \in G_y$, entonces $a y = y$. Sabemos que $y = g x$, entonces
    \[
        y = g x = g a x = g a (g^{-1} g) x = g a g^{-1} y
    \]
    Por tanto $g a g^{-1}$ fija a $y$, y así $g G_x g^{-1} \leq G_y$.

    \item[$(\supseteq)$]
    Si $a \in g G_x g^{-1}$, entonces $agxg^{-1} = gxg^{-1}$. Sabemos que $x = g^{-1} y$, entonces
    \[
        x = g^{-1} y = (agyg^{-1})^{-1} y = g^{-1} a g x 
    \]
    Por tanto $g^{-1} a g$ fija a $x$, y así $G_y \leq g G_x g^{-1}$.
\end{itemize}
Si $x$ y $y$ están en la misma órbita, entonces hay un $g \in G$ con $y = gx$ y de ahí
\[
    |G_y| = |G_{gx}| = |g G_y g^{-1}| = |G_x|
\]
\end{proof}


\subsection*{Ejercicio 2. (25 puntos)}
Considere a $G$ actuando en sí mismo por traslación. Es decir, tenemos la acción $\alpha : G \times G \to G$
dada por $\alpha(g, h) = gh$, donde el lado izquierdo es el producto del grupo $G$. Muestre que esta acción
tiene una sóla órbita y que $G_x = 1$ para todo $x \in G$.

\begin{proof}
Si $x \in G$, entonces la órbita $\mathcal{O}(x) = G$ porque si $g \in G$, entonces $g = (gx^{-1})x$. El
estabilizador $G_x$ de $x$ es $\{ 1 \}$ porque si $x = \alpha(a, x) = ax$, entonces $a = 1$. De hecho uno dicee
$G$ actúa transitivamente en $X$ cuando haya alguna $x \in X$ con $\mathcal{O}(x) = X$.
\end{proof}


\subsection*{Ejercicio 3. (25 puntos)}
Si $G$ es finito y $c$ es el número de clases de conjugación en $G$, demuestre que

$$ c = \frac{1}{|G|} \sum_{\tau \in G} |C_{G} (\tau)| $$

\begin{proof}
:(
\end{proof}


\subsection*{Ejercicio 4. (25 puntos)}
Sea $p$ un número primo y sea $X$ un $G$-conjunto finito, donde $|G| = p^n$ y $|X|$ no es divisible por $p$.
Demuestre que existe $x \in X$ tal que $\tau x = x$ para todo $\tau \in G$.

\begin{proof}
:(
\end{proof}

%%%%%%%%%%%%%%%%%%%%%%%%%%%%%%%%%%%%%%%%%%%%%%%%%%%%%%%%%%%%%%%%%%%%%%%%%%%%%%%%%%%%%%%%%


%%%%%%%%%%%%%%%%%%%%%%%%%%%%%%%%%%%%%%%%%%%%%%%%%%%%%%%%%%%%%%%%%%%%%%%%%%%%%%%%%%%%%%%%%

%%%%%%%%%%%%%%%%%%%%%%%%%%%%%%%%%%%%%%%%%%%%%%%%%%%%%%%%%%%%%%%%%%%%%%%%%%%%%%%%%%%%%%%%%

\end{document} 