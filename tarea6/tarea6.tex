%%%
 %
 % Copyright (C) 2020 Ángel Iván Gladín García
 %
 % This program is free software: you can redistribute it and/or modify
 % it under the terms of the GNU General Public License as published by
 % the Free Software Foundation, either version 3 of the License, or
 % (at your option) any later version.
 %
 % This program is distributed in the hope that it will be useful,
 % but WITHOUT ANY WARRANTY; without even the implied warranty of
 % MERCHANTABILITY or FITNESS FOR A PARTICULAR PURPOSE.  See the
 % GNU General Public License for more details.
 %
 % You should have received a copy of the GNU General Public License
 % along with this program.  If not, see <http://www.gnu.org/licenses/>.
%%%

%%%%%%%%%%%%%%%%%%%%%%%%%%%%%%%%%%%%%%%%%%%%%%%%%%%%%%%%%%%%%%%%%%%%%%%%%%%%%%%%%%%%%%%%%
\documentclass[12pt,letterpaper]{article}
\usepackage[margin=.75in]{geometry}
\usepackage[utf8]{inputenc}
\usepackage[spanish]{babel}
\decimalpoint

\usepackage{listings}
\usepackage{color}
\usepackage{graphicx}
\usepackage{enumerate}
\usepackage{enumitem}
\usepackage{float}

\usepackage{longtable}
\usepackage{hyperref}
\usepackage{commath}

\usepackage{bbm}
\usepackage{dsfont}
\usepackage{mathrsfs}
\usepackage{amsmath,amsthm,amssymb}
\usepackage{mathtools}
\usepackage{longtable}

\usepackage{tikz}
\usetikzlibrary{trees}
\usepackage{verbatim}

%%%%%%%%%%%%%%%%%%%%%%%%%%%%%%%%%%%%%%%%%%%%%%%%%%%%%%%%%%%%%%%%%%%%%%%%%%%%%%%%%%%%%%%%%%%%%%%%5

\usepackage{import}

\usepackage[utf8]{inputenc}

\usepackage{listings}
\usepackage{color}

%%%%%%%%%%%%%%%%%%%%%%%%%%%%%%%%%%%%%%%%%%%%%%%%%%%%%%%%%%%%%%%%%%%%%%%%%%%%%%%%%%%%%%%%%


%%%%%%%%%%%%%%%%%%%%%%%%%%%%%%%%%%%%%%%%%%%%%%%%%%%%%%%%%%%%%%%%%%%%%%%%%%%%%%%%%%%%%%%%%
\newcommand{\Z}{\mathbb{Z}}
\newcommand{\N}{\mathbb{N}}
\newcommand{\Q}{\mathbb{Q}}
\newcommand{\R}{\mathbb{R}}
\newcommand{\Pro}{\mathds{P}}
\newcommand{\Oh}{\mathcal{O}} %% Notacion "O"
\newcommand{\lra}{\longrightarrow}
\newcommand{\ra}{\rightarrow}
\newcommand{\ord}{\text{ord}}
\newcommand{\sol}{\textbf{\underline{Solución}: }} %% Solucion
\newcommand{\af}{\textbf{\underline{Afirmación}: }}
\newcommand{\cej}{\textbf{\underline{Contraejemplo}: }}

%%%%%%%%%%%%%%%%%%%%%%%%%%%%%%%%%%%%%%%%%%%%%%%%%%%%%%%%%%%%%%%%%%%%%%%%%%%%%%%%%%%%%%%%%

\begin{document}

%%%%%%%%%%%%%%%%%%%%%%%%%%%%%%%%%%%%%%%%%%%%%%%%%%%%%%%%%%%%%%%%%%%%%%%%%%%%%%%%%%%%%%%%%
\title{
    \vspace{-2.2em}
        Universidad Nacional Autónoma de México\\
        Facultad de Ciencias\\
        Álgebra Moderna I\\
    \vspace{.5cm}
    \large
        \textbf{Tarea 6}
}
\author{
    Ángel Iván Gladín García\\
    No. cuenta: 313112470\\
    \texttt{angelgladin@ciencias.unam.mx}
}
\date{3 de Marzo 2019}
\maketitle
%%%%%%%%%%%%%%%%%%%%%%%%%%%%%%%%%%%%%%%%%%%%%%%%%%%%%%%%%%%%%%%%%%%%%%%%%%%%%%%%%%%%%%%%%

%%%%%%%%%%%%%%%%%%%%%%%%%%%%%%%%%%%%%%%%%%%%%%%%%%%%%%%%%%%%%%%%%%%%%%%%%%%%%%%%%%%%%%%%%
\newtheorem{theorem}{Teorema}
\newtheorem{example}{Ejemplo}
\newtheorem{corollary}{Corolario}
\newtheorem{lemma}{Lemma}
\newtheorem{definition}{Definicion}
\newtheorem{prop}{Proposicion}
%%%%%%%%%%%%%%%%%%%%%%%%%%%%%%%%%%%%%%%%%%%%%%%%%%%%%%%%%%%%%%%%%%%%%%%%%%%%%%%%%%%%%%%%%

%%%%%%%%%%%%%%%%%%%%%%%%%%%%%%%%%%%%%%%%%%%%%%%%%%%%%%%%%%%%%%%%%%%%%%%%%%%%%%%%%%%%%%%%%
\subsection*{Ejercicio 1. (40 puntos)}
Demuestre que si $\alpha \in S_n$ es un $n$-ciclo, entonces su centralizador es $\langle a \rangle$.

\begin{proof}

Sea $\alpha = (1 \ 2 \ \ldots \ n)$ una permutación en $S_n$, entonces hay exactamente $(n-1)!$ permutaciones
que son iguales (porque $(1 \ 2 \ \ldots \ n) = (2 \ \ldots \ n \ 1) = \cdots 
= (n \ 1 \  \ldots \ n-1)$) y además como $|S_n| = n!$ entonces $\frac{n!}{n} = (n-1)!$).

Complementando lo anterior, sea $\beta \in S_n$ y $\alpha$ como lo menciona el inciso, se cumple que
$\beta \alpha \beta = \alpha$.

Como el número de conjugados de $\alpha$ denotado como $\alpha^{S_n}$ es igual a
$[S_n : C_{S_n}(\alpha)]$, entonces $|C_{S_n}(\alpha)| = n$.
Como $\langle \alpha \rangle \subseteq C_{S_n}(\alpha)$ y $\langle \alpha \rangle = n$,
ergo $\langle \alpha \rangle = C_{S_n}(\alpha)$.
\end{proof}

\subsection*{Ejercicio 2. (30 puntos)}
Demuestre que si $G$ no es abeliano, entonces $G/Z(G)$ no es cíclico.
\begin{proof} Por contrapositiva, es decir, \emph{Si $G/Z(G)$ es cíclico, entonces $G$ es abeliano}.

Si $G/Z(G)$ es cíclico con un generador $xZ(G)$ cada elemento en $G/Z(G)$ pueden ser escritos como
$x^k z$ para algún $k \in \Z$ y $z \in Z(G)$. Ahora sean $g, h \in G$, entonces $g = x^a z$ y
$h = x^b w$ para $z, w \in Z(G)$. Teniendo así que
$gh = x^a z x^b w = x^{a+b} zw = x^{b+a}wz = x^b w x^a z = hg$
\end{proof}

\subsection*{Ejercicio 3. (30 puntos)}
Demuestre que $Z(G_1 \times \cdots \times G_n) = Z(G_1) \times \cdots \times Z(G_n)$.
\begin{proof}
Sean $z, x \in G_1 \times \cdots \times G_n$ donde $z_i, x_i \in G_i$ para $1 \leq i \leq n$.
Se tiene que $Z(G_1 \times \cdots \times G_n) = \{ z \in G_i \ | \ zx = xz \quad \forall x \in G_i \}$.
Además $z \in Z(G_i)$ $\iff$ $zx = xz$
\begin{align*}
    &\iff (z_1, z_2, \ldots, z_n)(x_1, x_2, \ldots, x_n) = (x_1, x_2, \ldots, x_n)(z_1, z_2, \ldots , z_n)\\
    &\iff (z_1x_1, z_2x_2, \ldots, z_nx_n) = (x_1z_1, x_2z_2, \ldots, x_nz_n)\\
    &\iff z_ix_i = x_iz_i\\
    &\iff z_i \in Z(G_i)\\
    &\iff z \in Z(G_1) \times \cdots \times Z(G_n)
\end{align*}
\end{proof}



%%%%%%%%%%%%%%%%%%%%%%%%%%%%%%%%%%%%%%%%%%%%%%%%%%%%%%%%%%%%%%%%%%%%%%%%%%%%%%%%%%%%%%%%%


%%%%%%%%%%%%%%%%%%%%%%%%%%%%%%%%%%%%%%%%%%%%%%%%%%%%%%%%%%%%%%%%%%%%%%%%%%%%%%%%%%%%%%%%%

%%%%%%%%%%%%%%%%%%%%%%%%%%%%%%%%%%%%%%%%%%%%%%%%%%%%%%%%%%%%%%%%%%%%%%%%%%%%%%%%%%%%%%%%%

\end{document}