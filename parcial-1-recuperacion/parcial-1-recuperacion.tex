%%%
 %
 % Copyright (C) 2020 Ángel Iván Gladín García
 %
 % This program is free software: you can redistribute it and/or modify
 % it under the terms of the GNU General Public License as published by
 % the Free Software Foundation, either version 3 of the License, or
 % (at your option) any later version.
 %
 % This program is distributed in the hope that it will be useful,
 % but WITHOUT ANY WARRANTY; without even the implied warranty of
 % MERCHANTABILITY or FITNESS FOR A PARTICULAR PURPOSE.  See the
 % GNU General Public License for more details.
 %
 % You should have received a copy of the GNU General Public License
 % along with this program.  If not, see <http://www.gnu.org/licenses/>.
%%%

%%%%%%%%%%%%%%%%%%%%%%%%%%%%%%%%%%%%%%%%%%%%%%%%%%%%%%%%%%%%%%%%%%%%%%%%%%%%%%%%%%%%%%%%%
\documentclass[letterpaper]{article}
\usepackage[margin=.75in]{geometry}
\usepackage[utf8]{inputenc}
\usepackage[spanish]{babel}
\decimalpoint

\usepackage{listings}
\usepackage{color}
\usepackage{graphicx}
\usepackage{enumerate}
\usepackage{enumitem}
\usepackage{float}

\usepackage{longtable}
\usepackage{hyperref}
\usepackage{commath}

\usepackage{bbm}
\usepackage{dsfont}
\usepackage{mathrsfs}
\usepackage{amsmath,amsthm,amssymb}
\usepackage{mathtools}
\usepackage{longtable}

\usepackage{tikz}
\usetikzlibrary{trees}
\usepackage{verbatim}
\usepackage{wasysym}

%%%%%%%%%%%%%%%%%%%%%%%%%%%%%%%%%%%%%%%%%%%%%%%%%%%%%%%%%%%%%%%%%%%%%%%%%%%%%%%%%%%%%%%%%%%%%%%%5

\usepackage{import}

\usepackage[utf8]{inputenc}

\usepackage{listings}
\usepackage{color}

%%%%%%%%%%%%%%%%%%%%%%%%%%%%%%%%%%%%%%%%%%%%%%%%%%%%%%%%%%%%%%%%%%%%%%%%%%%%%%%%%%%%%%%%%


%%%%%%%%%%%%%%%%%%%%%%%%%%%%%%%%%%%%%%%%%%%%%%%%%%%%%%%%%%%%%%%%%%%%%%%%%%%%%%%%%%%%%%%%%
\newcommand{\Z}{\mathbb{Z}}
\newcommand{\N}{\mathbb{N}}
\newcommand{\Q}{\mathbb{Q}}
\newcommand{\R}{\mathbb{R}}
\newcommand{\Pro}{\mathds{P}}
\newcommand{\Oh}{\mathcal{O}} %% Notacion "O"
\newcommand{\lra}{\longrightarrow}
\newcommand{\ra}{\rightarrow}
\newcommand{\ord}{\text{ord}}
\newcommand{\sol}{\textbf{\underline{Solución}: }} %% Solucion
\newcommand{\af}{\textbf{\underline{Afirmación}: }}
\newcommand{\cej}{\textbf{\underline{Contraejemplo}: }}


%%%%%%%%%%%%%%%%%%%%%%%%%%%%%%%%%%%%%%%%%%%%%%%%%%%%%%%%%%%%%%%%%%%%%%%%%%%%%%%%%%%%%%%%%

\begin{document}

%%%%%%%%%%%%%%%%%%%%%%%%%%%%%%%%%%%%%%%%%%%%%%%%%%%%%%%%%%%%%%%%%%%%%%%%%%%%%%%%%%%%%%%%%
\title{
    \vspace{-2.2em}
        Universidad Nacional Autónoma de México\\
        Facultad de Ciencias\\
        Álgebra Moderna I\\
    \vspace{.5cm}
    \large
        \textbf{Primer examen parcial de recuperación}
}
\author{
    Ángel Iván Gladín García\\
    No. cuenta: 313112470\\
    \texttt{angelgladin@ciencias.unam.mx}
}
\date{3 de junio de 2020}
\maketitle
%%%%%%%%%%%%%%%%%%%%%%%%%%%%%%%%%%%%%%%%%%%%%%%%%%%%%%%%%%%%%%%%%%%%%%%%%%%%%%%%%%%%%%%%%

%%%%%%%%%%%%%%%%%%%%%%%%%%%%%%%%%%%%%%%%%%%%%%%%%%%%%%%%%%%%%%%%%%%%%%%%%%%%%%%%%%%%%%%%%
\newtheorem{theorem}{Teorema}
\newtheorem{example}{Ejemplo}
\newtheorem{corollary}{Corolario}
\newtheorem{lemma}{Lemma}
\newtheorem{definition}{Definicion}
\newtheorem{prop}{Proposicion}
%%%%%%%%%%%%%%%%%%%%%%%%%%%%%%%%%%%%%%%%%%%%%%%%%%%%%%%%%%%%%%%%%%%%%%%%%%%%%%%%%%%%%%%%%

%%%%%%%%%%%%%%%%%%%%%%%%%%%%%%%%%%%%%%%%%%%%%%%%%%%%%%%%%%%%%%%%%%%%%%%%%%%%%%%%%%%%%%%%%
\subsection*{Ejercicio 1. (25 puntos)}
Pruebe que si $\alpha = \beta_1 \beta_2 \cdots \beta_m$ es un producto de $r_i$-ciclos disjuntos $\beta_i$,
entonces el entero positivo $s$ más pequeño con $\alpha^s = 1$ es el mínimo cómun múltiplo de
$\{ r_1, r_2, \ldots, r_m \}$.

\begin{proof}
Como por hipótesis los $\beta_i$ son disjuntos, podemos usar el hecho de que conmutan para poder hacer lo siguiente,
\[
    \alpha^n = (\beta_1 \beta_2 \cdots \beta_m)^n = \beta_1^n \beta_2^n \cdots \beta_m^n
\]
para toda $n \in \Z$. Supongamos que $\alpha^n = 1$ si y sólo si $\beta_i^n = 1$ para $1 \leq i \leq m$.
Como $\alpha$ es un $n$-ciclo y $\beta_i$ es un $r_i$-ciclo, se sigue que que todo $r_i$ divide a $n$, y éso
a su vez implica que el $m.c.m.(r_1, r_2, \ldots, r_m) \mid n$. Como $\beta_i^n = 1$ para $1 \leq i \leq m$ 
justo aseguramos que $n$ es el entero más pequeño. Renombrando a $m.c.m.(r_1, r_2, \ldots, r_m) = s = n$,
se sigue que $s$ es el entero mas pequeño positivo tal que $\alpha^s = 1$.
\end{proof}

\subsection*{Ejercicio 2. (25 puntos)}
Sea $S$ un subconjunto no vacío de un grupo $G$ y definimos una relación en $G$ dado por $a \sim b$ si y
sólo si $ab^{-1} \in S$. Muestre que $\sim$ es una relación de equivalencia si y sólo si $S$ es un subgrupo
de $G$.

\begin{proof}
\hfill
\begin{itemize}
    \item \textbf{Reflexividad:} Como $aa^{-1} = 1 \in S$, por tanto $a \sim a$.

    \item \textbf{Simetría:} Se tiene que $a \sim b$, entonces $ab^{-1} \in S$, entonces $(ab^{-1})^{-1} \in S$,
    de ahí que $ba^{-1} \in S$. Por tanto $b \sim a$.

    \item \textbf{Transitividad:} Si $a \sim b$ y $b \sim c$, se tiene que $ab^{-1} \in S$ y $bc^{-1} \in S$, pero se
    tiene que $(ab^{-1})(bc^{-1}) \in S$, lo cual implica que $ac^{-1} \in S$. Por tanto $a \sim c$.
\end{itemize}

Se asumió que $S$ era un subgrupo de $G$ por un teorema\footnote{
    \textbf{Teorema 2.2} \emph{Un subconjunto $S$ de un grupo $G$ es un subgrupo si y sólo si $1 \in S$ y
    $s, t \in S$ implica que $st^{-1} \in S$}
} visto en clase.
\end{proof}

\subsection*{Ejercicio 3. (25 puntos)}
Demuestra el Teorema de Wilson: Si $p$ es un primo, entonces
\[
    (p - 1)! \equiv -1 \mod p
\]

\begin{proof}
Consideremos el grupo $G$ de elementos no cero de $\Z_p$ forma un grupo multiplicativo como
$G = ((\Z/p\Z)^\times, \cdot)$ donde $(\Z/p\Z)^\times = \{ [1], [2], \ldots, [p-1] \}$ y $(\cdot)$
como el producto en $\Z$, también se tiene que $G$ es abeliano por $\Z$.

Observemos que como $p$ es primo y sea $x \in (\Z/p\Z)^\times$ tal que $x^2 = 1$ si tiene que:
\begin{align*}
    &\qquad\quad x^2 \equiv 1 \mod p\\
    &\iff x^2 -1 \equiv 0 \mod p\\
    &\iff (x+1)(x-1) \equiv 0 \mod p\\
    &x = np \pm 1
\end{align*}
Concluyendo así que $x = 1$ ó $x = p-1$.

Tambien observando que se cumple que $(p-2)! \equiv 1 \mod p$.

Por tanto,
\[
    (p-1)! \equiv (p-2)! \cdot (p-1) \equiv -1 \mod p
\]
\end{proof}

\subsection*{Ejercicio 4. (25 puntos)}
Un grupo $G$ es abeliano si y sólo si la función $\varphi : G \to G$ dada por $\varphi(g) = g^{-1}$
es un automorfismo\footnote{Un automorfismo de un grupo $G$ es un isomorfismo $\varphi : G \to G$.}.

\begin{proof}
\hfill
\begin{itemize}
    \item[$(\Longrightarrow)$] Supongamos que $G$ es abeliano. Sean $g, h \in G$. Si $\varphi(g) = \varphi(h)$,
    entonces se tiene que $g^{-1} = h^{-1}$, lo que implica que $g = h$ provando así que $\varphi$ es inyectiva.

    Sea $g \in G$, se tiene que $g^{-1} \in G$ y $\varphi(g^{-1}) = (g^{-1})^{-1} = g$ teniendo así que $\varphi$ es
    suprayectiva.

    Para concluir, sean $g, h \in G$ se sigue que:
    \begin{align*}
        \varphi(gh)
            &= (gh)^{-1} = h^{-1}g^{-1}\\
            &= g^{-1}h^{-1} && \text{Por ser $G$ abeliano.}\\
            &= \varphi(g)\varphi(h)
    \end{align*}
    Teniendo así que $\varphi$ es un automorfismo.

    \item[$(\Longleftarrow)$] Supongamos que la función $\varphi : G \to G$ dada por $\varphi(g) = g^{-1}$
    es un automorfismo. Sean $g, h \in G$, entonces:
    \begin{align*}
        h^{-1}g^{-1}
            &= (gh)^{-1}\\
            &= \varphi(gh)\\
            &= \varphi(g)\varphi(h) && \text{Por ser $\varphi$ un homomorfismo.}\\
            &= g^{-1}h^{-1}
    \end{align*}
    Lo que implica que $gh = hg$. Como $g, h \in G$ fueron arbitrarios, se concluye que $G$ es abeliano.
\end{itemize}
\end{proof}


%%%%%%%%%%%%%%%%%%%%%%%%%%%%%%%%%%%%%%%%%%%%%%%%%%%%%%%%%%%%%%%%%%%%%%%%%%%%%%%%%%%%%%%%%


%%%%%%%%%%%%%%%%%%%%%%%%%%%%%%%%%%%%%%%%%%%%%%%%%%%%%%%%%%%%%%%%%%%%%%%%%%%%%%%%%%%%%%%%%

%%%%%%%%%%%%%%%%%%%%%%%%%%%%%%%%%%%%%%%%%%%%%%%%%%%%%%%%%%%%%%%%%%%%%%%%%%%%%%%%%%%%%%%%%

\end{document} 