%%%
 %
 % Copyright (C) 2020 Ángel Iván Gladín García
 %
 % This program is free software: you can redistribute it and/or modify
 % it under the terms of the GNU General Public License as published by
 % the Free Software Foundation, either version 3 of the License, or
 % (at your option) any later version.
 %
 % This program is distributed in the hope that it will be useful,
 % but WITHOUT ANY WARRANTY; without even the implied warranty of
 % MERCHANTABILITY or FITNESS FOR A PARTICULAR PURPOSE.  See the
 % GNU General Public License for more details.
 %
 % You should have received a copy of the GNU General Public License
 % along with this program.  If not, see <http://www.gnu.org/licenses/>.
%%%

%%%%%%%%%%%%%%%%%%%%%%%%%%%%%%%%%%%%%%%%%%%%%%%%%%%%%%%%%%%%%%%%%%%%%%%%%%%%%%%%%%%%%%%%%
\documentclass[letterpaper]{article}
\usepackage[margin=.75in]{geometry}
\usepackage[utf8]{inputenc}
\usepackage[spanish]{babel}
\decimalpoint

\usepackage{listings}
\usepackage{color}
\usepackage{graphicx}
\usepackage{enumerate}
\usepackage{enumitem}
\usepackage{float}

\usepackage{longtable}
\usepackage{hyperref}
\usepackage{commath}

\usepackage{bbm}
\usepackage{dsfont}
\usepackage{mathrsfs}
\usepackage{amsmath,amsthm,amssymb}
\usepackage{mathtools}
\usepackage{longtable}

\usepackage{tikz}
\usetikzlibrary{trees}
\usepackage{verbatim}

%%%%%%%%%%%%%%%%%%%%%%%%%%%%%%%%%%%%%%%%%%%%%%%%%%%%%%%%%%%%%%%%%%%%%%%%%%%%%%%%%%%%%%%%%%%%%%%%5

\usepackage{import}

\usepackage[utf8]{inputenc}

\usepackage{listings}
\usepackage{color}

%%%%%%%%%%%%%%%%%%%%%%%%%%%%%%%%%%%%%%%%%%%%%%%%%%%%%%%%%%%%%%%%%%%%%%%%%%%%%%%%%%%%%%%%%


%%%%%%%%%%%%%%%%%%%%%%%%%%%%%%%%%%%%%%%%%%%%%%%%%%%%%%%%%%%%%%%%%%%%%%%%%%%%%%%%%%%%%%%%%
\newcommand{\Z}{\mathbb{Z}}
\newcommand{\N}{\mathbb{N}}
\newcommand{\Q}{\mathbb{Q}}
\newcommand{\R}{\mathbb{R}}
\newcommand{\Pro}{\mathds{P}}
\newcommand{\Oh}{\mathcal{O}} %% Notacion "O"
\newcommand{\lra}{\longrightarrow}
\newcommand{\ra}{\rightarrow}
\newcommand{\ord}{\text{ord}}
\newcommand{\sol}{\textbf{\underline{Solución}: }} %% Solucion
\newcommand{\af}{\textbf{\underline{Afirmación}: }}
\newcommand{\cej}{\textbf{\underline{Contraejemplo}: }}

%%%%%%%%%%%%%%%%%%%%%%%%%%%%%%%%%%%%%%%%%%%%%%%%%%%%%%%%%%%%%%%%%%%%%%%%%%%%%%%%%%%%%%%%%

\begin{document}

%%%%%%%%%%%%%%%%%%%%%%%%%%%%%%%%%%%%%%%%%%%%%%%%%%%%%%%%%%%%%%%%%%%%%%%%%%%%%%%%%%%%%%%%%
\title{
    \vspace{-2.2em}
        Universidad Nacional Autónoma de México\\
        Facultad de Ciencias\\
        Álgebra Moderna I\\
    \vspace{.5cm}
    \large
        \textbf{Tarea 7}
}
\author{
    Ángel Iván Gladín García\\
    No. cuenta: 313112470\\
    \texttt{angelgladin@ciencias.unam.mx}
}
\date{20 de Marzo 2020}
\maketitle
%%%%%%%%%%%%%%%%%%%%%%%%%%%%%%%%%%%%%%%%%%%%%%%%%%%%%%%%%%%%%%%%%%%%%%%%%%%%%%%%%%%%%%%%%

%%%%%%%%%%%%%%%%%%%%%%%%%%%%%%%%%%%%%%%%%%%%%%%%%%%%%%%%%%%%%%%%%%%%%%%%%%%%%%%%%%%%%%%%%
\newtheorem{theorem}{Teorema}
\newtheorem{example}{Ejemplo}
\newtheorem{corollary}{Corolario}
\newtheorem{lemma}{Lemma}
\newtheorem{definition}{Definicion}
\newtheorem{prop}{Proposicion}
%%%%%%%%%%%%%%%%%%%%%%%%%%%%%%%%%%%%%%%%%%%%%%%%%%%%%%%%%%%%%%%%%%%%%%%%%%%%%%%%%%%%%%%%%

%%%%%%%%%%%%%%%%%%%%%%%%%%%%%%%%%%%%%%%%%%%%%%%%%%%%%%%%%%%%%%%%%%%%%%%%%%%%%%%%%%%%%%%%%
\subsection*{Ejercicio 1. (50 puntos)}
Si $n \neq 4$, demuestre que $A_n$ es el único subgrupo normal propio no trivial de $S_n$.

\begin{proof}
El grupo $A_3$ puede verse fácilmente que es simple. Cuando $n = 4$ tenemos el subgrupo normal
$\{ 1, (12)(34), (13)(24), (14)(23) \}$ en $A_4$, teniendo así que $A_4$ no es simple.

Asumamos que $n \geq 5$.

\textbf{Paso 1} $A_n$ es generado por 3-ciclcos.

De hecho cualquier elemento de $A_n$ es un producto de transposiciones de la forma $(ab)(cd)$ o $(ab)(ac)$.
Como $(ab)(cd) = (acb)(acd)$ y $(ab)(ac)=(acb)$ concluimos que $A_n$ es generado por los 3-ciclos.
Ademas $(1a2)=(12a)^{-1}$, $(1ab)=(12b)(12a)^{-1}$, $(2ab)=(12b)^{-1}(12a)$, y
$(abc)=(12a)^{-1}(12c)(12b)^{-1}(12a)$, los que muestra que cada 3-ciclo es generado por un ciclo de la
forma $(12k)$.


\textbf{Paso 2} Si $H$ es un subgrupo normal de $A_n$. $H$ contiene un 3-ciclo, entonces $H = A_n$.

Sin pérdida de generalidad $(123) \in H$, Entonces
$(12k) = ((12)(3k))(123)^{-1}((12)(3k))^{-1} = ((123)^{-1})^{(12)(3k)} \in H$ por normalidad. Así
$A_n = \langle (12k) \ : \ k \geq 3 \rangle \leq H$, y $H = A_n$.

Supongamos que $H \neq 1$ es normal en $A_n$, entonces debemos exhibir que $H$ necesariamente contiene un
3-ciclo. Luego usamos el paso 2 para concluir la demostración. Por casos:
\begin{enumerate}
    \item Sin perdida de generalidad suponer que $H$ contiene a $\sigma = (1 2 \cdots r)\tau$, donde
    $r \geq 4$, y $\tau$ es disjunto de $\{ 1,2,\ldots,r \}$. Entonces por normalidad de $H$, $H$ contiene
    $\sigma^{-1}\sigma^{(123)} = (12r)$. Y acabamos con el paso 2.

    \item Supongamos sin perdida de generalidad $\sigma = (123)(456)\tau \in H$, donde $\tau$ es el
    producto de transposiciones disjuntas. Entonces $H$ contiene $\sigma^{-1}\sigma^{(124)} = (14263) = (12r)$
    y acabamos con el paso 1.

    \item Supongamos sin perdida de generalidad que $(123)\tau \in H$, con $\tau$ un producto de
    transposiciones disjuntas, entonces $H$ contiene a $(123)\tau(123)\tau = (132)$, y acabamos con el paso 2.

    \item Supongamos que $H$ contiene elementos que son productos de transposiciones disjuntas. Sin pérdida de
    generalidad sea $(12)(34)\tau$ sea un elemento de $H$. Entonces
    $(12)(34)\tau((12)(34)\tau)^{(123)} = (13)(24) \in H$. Como $n \geq 5$, consideremos $(123) \in A_n$. Por
    normalidad $H$ contiene a $(13)(24)((13)(24))^{(135)} = (135)$ y acabamos con el paso 2.
\end{enumerate}


\end{proof}

\subsection*{Ejercicio 2. (50 puntos)}
Demuestre que si $G \leq S_n$ contiene una permutación impar, entonces $|G|$ es par y exactamente la mitad
de los elementos de $G$ son permutaciones impares.

\begin{proof}
Supongamos que $G$ es un subgrupo de $S_n$. Notemos que $G$ contiene a la permutación identidad, la cual es
par, así $G$ no consiste de entéramente de permutaciones impares. Si pasara que todos los elementos de $G$
son pares, entonces ya acabaríamos.

En consecuenciua supongamos algunos elementos de $G$ son par y otros impar. Sea $E$ el conjunto de
permutaciones pares en $G$, y sea $O$ el conjunto de permutaciones impares. Ahora queremos demostrar que
$E$ y $O$ tienen la misma cardinalidad, lo que significa que queremos exhibir una biyección $\varphi : E \to O$.

Escojamos una permutación impar $\sigma \in O$, y definamos $\varphi$ por la regla $\varphi(x) = \sigma x$.
Notemos que esta función tiene sentido. Si $x \in E$, entonces $x$ es par, y como $\sigma$ es impar,
$\varphi(x) = \sigma x$ es impar. Además, como $x$ y $\sigma$ están en $G$, entonces $\varphi(x)=\sigma x$
están en $G$ también porque $G$ es cerrado. Por consiguiente, $\varphi$ manda permutaciones pares en $G$ a
permutaciones impares en $G$.

Para ver que $\varphi$ es uno a uno, supongamos $\varphi(\tau) = \varphi(\mu)$. Que significa
$\sigma \tau = \sigma \mu$, entonces $\tau = \mu$ por cancelación en $S_n$. Por tanto $\varphi$ es uno
a uno.

Para ver que es suprayectiva, sea $\mu$ una permutación arbitraria en $O$. Entonces $\sigma^{-1}\mu$ es una
permutación par, y está en $G$ porque ambas $\sigma$ y $\mu$ están en $G$. Por consiguiente
$\sigma^{-1}\mu \in E$. Observemos que $\varphi(\sigma^{-1}\mu) = \sigma \sigma^{-1} \mu = \mu$ y se sigue
que $\varphi$ es suprayectiva.

Esto completa la demostración de que hay una función biyectiva $\varphi : E \to O$, así $|E| = |O|$. Por tanto
la mitad de las permutaciones de $G$ son pares y la otra mitad son impares.
\end{proof}
%%%%%%%%%%%%%%%%%%%%%%%%%%%%%%%%%%%%%%%%%%%%%%%%%%%%%%%%%%%%%%%%%%%%%%%%%%%%%%%%%%%%%%%%%


%%%%%%%%%%%%%%%%%%%%%%%%%%%%%%%%%%%%%%%%%%%%%%%%%%%%%%%%%%%%%%%%%%%%%%%%%%%%%%%%%%%%%%%%%

%%%%%%%%%%%%%%%%%%%%%%%%%%%%%%%%%%%%%%%%%%%%%%%%%%%%%%%%%%%%%%%%%%%%%%%%%%%%%%%%%%%%%%%%%

\end{document} 