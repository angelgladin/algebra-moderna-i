%%%
 %
 % Copyright (C) 2020 Ángel Iván Gladín García
 %
 % This program is free software: you can redistribute it and/or modify
 % it under the terms of the GNU General Public License as published by
 % the Free Software Foundation, either version 3 of the License, or
 % (at your option) any later version.
 %
 % This program is distributed in the hope that it will be useful,
 % but WITHOUT ANY WARRANTY; without even the implied warranty of
 % MERCHANTABILITY or FITNESS FOR A PARTICULAR PURPOSE.  See the
 % GNU General Public License for more details.
 %
 % You should have received a copy of the GNU General Public License
 % along with this program.  If not, see <http://www.gnu.org/licenses/>.
%%%

%%%%%%%%%%%%%%%%%%%%%%%%%%%%%%%%%%%%%%%%%%%%%%%%%%%%%%%%%%%%%%%%%%%%%%%%%%%%%%%%%%%%%%%%%
\documentclass[letterpaper]{article}
\usepackage[margin=.75in]{geometry}
\usepackage[utf8]{inputenc}
\usepackage[spanish]{babel}
\decimalpoint

\usepackage{listings}
\usepackage{color}
\usepackage{graphicx}
\usepackage{enumerate}
\usepackage{enumitem}
\usepackage{float}

\usepackage{longtable}
\usepackage{hyperref}
\usepackage{commath}

\usepackage{bbm}
\usepackage{dsfont}
\usepackage{mathrsfs}
\usepackage{amsmath,amsthm,amssymb}
\usepackage{mathtools}
\usepackage{longtable}

\usepackage{tikz}
\usetikzlibrary{trees}
\usepackage{verbatim}
\usepackage{wasysym}

%%%%%%%%%%%%%%%%%%%%%%%%%%%%%%%%%%%%%%%%%%%%%%%%%%%%%%%%%%%%%%%%%%%%%%%%%%%%%%%%%%%%%%%%%%%%%%%%5

\usepackage{import}

\usepackage[utf8]{inputenc}

\usepackage{listings}
\usepackage{color}

%%%%%%%%%%%%%%%%%%%%%%%%%%%%%%%%%%%%%%%%%%%%%%%%%%%%%%%%%%%%%%%%%%%%%%%%%%%%%%%%%%%%%%%%%


%%%%%%%%%%%%%%%%%%%%%%%%%%%%%%%%%%%%%%%%%%%%%%%%%%%%%%%%%%%%%%%%%%%%%%%%%%%%%%%%%%%%%%%%%
\newcommand{\Z}{\mathbb{Z}}
\newcommand{\N}{\mathbb{N}}
\newcommand{\Q}{\mathbb{Q}}
\newcommand{\R}{\mathbb{R}}
\newcommand{\Pro}{\mathds{P}}
\newcommand{\Oh}{\mathcal{O}} %% Notacion "O"
\newcommand{\lra}{\longrightarrow}
\newcommand{\ra}{\rightarrow}
\newcommand{\ord}{\text{ord}}
\newcommand{\sol}{\textbf{\underline{Solución}: }} %% Solucion
\newcommand{\af}{\textbf{\underline{Afirmación}: }}
\newcommand{\cej}{\textbf{\underline{Contraejemplo}: }}

%%%%%%%%%%%%%%%%%%%%%%%%%%%%%%%%%%%%%%%%%%%%%%%%%%%%%%%%%%%%%%%%%%%%%%%%%%%%%%%%%%%%%%%%%

\begin{document}

%%%%%%%%%%%%%%%%%%%%%%%%%%%%%%%%%%%%%%%%%%%%%%%%%%%%%%%%%%%%%%%%%%%%%%%%%%%%%%%%%%%%%%%%%
\title{
    \vspace{-2.2em}
        Universidad Nacional Autónoma de México\\
        Facultad de Ciencias\\
        Álgebra Moderna I\\
    \vspace{.5cm}
    \large
        \textbf{Tarea-examen \emph{Teoremas de Sylow}}
}
\author{
    Ángel Iván Gladín García\\
    No. cuenta: 313112470\\
    \texttt{angelgladin@ciencias.unam.mx}
}
\date{27 de Mayo 2020}
\maketitle
%%%%%%%%%%%%%%%%%%%%%%%%%%%%%%%%%%%%%%%%%%%%%%%%%%%%%%%%%%%%%%%%%%%%%%%%%%%%%%%%%%%%%%%%%

%%%%%%%%%%%%%%%%%%%%%%%%%%%%%%%%%%%%%%%%%%%%%%%%%%%%%%%%%%%%%%%%%%%%%%%%%%%%%%%%%%%%%%%%%
\newtheorem{theorem}{Teorema}
\newtheorem{example}{Ejemplo}
\newtheorem{corollary}{Corolario}
\newtheorem{lemma}{Lemma}
\newtheorem{definition}{Definicion}
\newtheorem{prop}{Proposicion}
%%%%%%%%%%%%%%%%%%%%%%%%%%%%%%%%%%%%%%%%%%%%%%%%%%%%%%%%%%%%%%%%%%%%%%%%%%%%%%%%%%%%%%%%%

%%%%%%%%%%%%%%%%%%%%%%%%%%%%%%%%%%%%%%%%%%%%%%%%%%%%%%%%%%%%%%%%%%%%%%%%%%%%%%%%%%%%%%%%%
\subsection*{Ejercicio 1. (25 puntos)}
Sea $H$ un subgrupo normal de $G$. Si ambos $H$ y $G/H$ son $p$-grupos, entonces $G$ es un $p$-grupo.
\begin{proof}
Sea $g \in G$ y sea $gH \in G/H$. Como $G/H$ por hipótesis es un $p$-grupo entonces\footnote{
    \textbf{Corolario 4.3} \emph{Un grupo finito $G$ es un $p$-grupo si y solo si $|G|$ es una potencia de $p$.}
} $|gH| = p^n$ para algún $n > 0$. Por tanto $(gH)^{p^n} = H$ y así $g^{p^n} \in H$.

Como $H$ es un $p$-grupo, para algún $m > 0$ se tiene que $\left(g^{p^n}\right)^{p^m} = g^{p^{n+m}}=1$, y así se
concluye que $G$ es un $p$-grupo.
\end{proof}

\subsection*{Ejercicio 2. (25 puntos)}
Demuestre que cualquier grupo de orden 200 contiene un subgrupo de Sylow normal.
\begin{proof}
Sea $|G| = 200 = 2^3 \cdot 5^2$. Por el terorema de Sylow\footnote{
    \textbf{Teorema 4.12 (Sylow) (ii)} \emph{Si hay $r$ $p$-subgrupos de Sylow, entonces $r \mid |G|$ y
    $r \equiv 1 \mod p$.}
} consideremos $r_p$ el número de de $p$-subgrupos de Sylow. Entonces $r_5 = 1$. Por tanto hay un único
5-subgrupo de Sylow $P$. Como cualquier conjugado de $P$ es también un conjugado de $P$ es también un
5-subgrupo de Sylow, se concluye que $gPg^{-1} = P$ para toda $g \in G$ y así $P \lhd G$.
\end{proof}

\subsection*{Ejercicio 3. (25 puntos)}
Si $P$ es un $p$-subgrupo de Sylow normal de un grupo finito $G$ y $f : G \to G$ es un morfismo de grupos,
entonces $f(P) < P$.
\begin{proof}
Notemos que $P \leq G$ y como la imagen de homomorfismos de un grupo también es un grupo, entonces $f(P) \leq G$.
Sea $y \in f(P)$. Entonces existe $x \in P$ tal que $f(x) = y$. Como $P$ es un $p$-subgrupo entonces
$|x| = p^i$ para alguna $i$. Entonces se tiene que $y^{p^i} = f(x)^{p^i} = f(x^{p^i}) = f(1) = 1$.
Entonces $|y|$ divide a $p^i$ y así $|y| = p^j$ para algún $j \leq i$. Como la $y$ fue arbitraria $f(P)$
es un $p$-subgrupo de $G$. Por el teorema de Sylow\footnote{
    \textbf{Teorema 4.12 (Sylow) (i)} \emph{Si $P$ es un $p$-subgrupo de Sylow de un subgrupo finito $G$,
    entonces todos los $p$-subgrupos de $G$ son conjugados de $P$.}
} entonces existe $x \in G$ tal que $f(P) \leq xPx^{-1} = P$.
\end{proof}

\subsection*{Ejercicio 4. (25 puntos)}
Si $Q$ es un $p$-subgrupo normal de un grupo finito $G$, entonces $Q \leq P$ para cualquier $p$-subgrupo
de Sylow $P$.
\begin{proof}
Por Sylow $Q$ está contenido en un $p$-subgrupo de Sylow $H$ de $G$. Por el teorema de Sylow, $P = xHx^{-1}$
para alguna $x \in G$. Así $Q = xQx^{-1} \leq xHx^{-1} = P$.
\end{proof}

%%%%%%%%%%%%%%%%%%%%%%%%%%%%%%%%%%%%%%%%%%%%%%%%%%%%%%%%%%%%%%%%%%%%%%%%%%%%%%%%%%%%%%%%%


%%%%%%%%%%%%%%%%%%%%%%%%%%%%%%%%%%%%%%%%%%%%%%%%%%%%%%%%%%%%%%%%%%%%%%%%%%%%%%%%%%%%%%%%%

%%%%%%%%%%%%%%%%%%%%%%%%%%%%%%%%%%%%%%%%%%%%%%%%%%%%%%%%%%%%%%%%%%%%%%%%%%%%%%%%%%%%%%%%%

\end{document} 