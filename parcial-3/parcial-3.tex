%%%
 %
 % Copyright (C) 2020 Ángel Iván Gladín García
 %
 % This program is free software: you can redistribute it and/or modify
 % it under the terms of the GNU General Public License as published by
 % the Free Software Foundation, either version 3 of the License, or
 % (at your option) any later version.
 %
 % This program is distributed in the hope that it will be useful,
 % but WITHOUT ANY WARRANTY; without even the implied warranty of
 % MERCHANTABILITY or FITNESS FOR A PARTICULAR PURPOSE.  See the
 % GNU General Public License for more details.
 %
 % You should have received a copy of the GNU General Public License
 % along with this program.  If not, see <http://www.gnu.org/licenses/>.
%%%

%%%%%%%%%%%%%%%%%%%%%%%%%%%%%%%%%%%%%%%%%%%%%%%%%%%%%%%%%%%%%%%%%%%%%%%%%%%%%%%%%%%%%%%%%
\documentclass[letterpaper]{article}
\usepackage[margin=.75in]{geometry}
\usepackage[utf8]{inputenc}
\usepackage[spanish]{babel}
\decimalpoint

\usepackage{listings}
\usepackage{color}
\usepackage{graphicx}
\usepackage{enumerate}
\usepackage{enumitem}
\usepackage{float}

\usepackage{longtable}
\usepackage{hyperref}
\usepackage{commath}

\usepackage{bbm}
\usepackage{dsfont}
\usepackage{mathrsfs}
\usepackage{amsmath,amsthm,amssymb}
\usepackage{mathtools}
\usepackage{longtable}

\usepackage{tikz}
\usetikzlibrary{trees}
\usepackage{verbatim}
\usepackage{wasysym}

%%%%%%%%%%%%%%%%%%%%%%%%%%%%%%%%%%%%%%%%%%%%%%%%%%%%%%%%%%%%%%%%%%%%%%%%%%%%%%%%%%%%%%%%%%%%%%%%5

\usepackage{import}

\usepackage[utf8]{inputenc}

\usepackage{listings}
\usepackage{color}

%%%%%%%%%%%%%%%%%%%%%%%%%%%%%%%%%%%%%%%%%%%%%%%%%%%%%%%%%%%%%%%%%%%%%%%%%%%%%%%%%%%%%%%%%


%%%%%%%%%%%%%%%%%%%%%%%%%%%%%%%%%%%%%%%%%%%%%%%%%%%%%%%%%%%%%%%%%%%%%%%%%%%%%%%%%%%%%%%%%
\newcommand{\Z}{\mathbb{Z}}
\newcommand{\N}{\mathbb{N}}
\newcommand{\Q}{\mathbb{Q}}
\newcommand{\R}{\mathbb{R}}
\newcommand{\Pro}{\mathds{P}}
\newcommand{\Oh}{\mathcal{O}} %% Notacion "O"
\newcommand{\lra}{\longrightarrow}
\newcommand{\ra}{\rightarrow}
\newcommand{\ord}{\text{ord}}
\newcommand{\sol}{\textbf{\underline{Solución}: }} %% Solucion
\newcommand{\af}{\textbf{\underline{Afirmación}: }}
\newcommand{\cej}{\textbf{\underline{Contraejemplo}: }}

%%%%%%%%%%%%%%%%%%%%%%%%%%%%%%%%%%%%%%%%%%%%%%%%%%%%%%%%%%%%%%%%%%%%%%%%%%%%%%%%%%%%%%%%%

\begin{document}

%%%%%%%%%%%%%%%%%%%%%%%%%%%%%%%%%%%%%%%%%%%%%%%%%%%%%%%%%%%%%%%%%%%%%%%%%%%%%%%%%%%%%%%%%
\title{
    \vspace{-2.75em}
        Universidad Nacional Autónoma de México\\
        Facultad de Ciencias\\
        Álgebra Moderna I\\
    \vspace{.5cm}
    \large
        \textbf{Tercer examen parcial}
}
\author{
    Ángel Iván Gladín García\\
    No. cuenta: 313112470\\
    \texttt{angelgladin@ciencias.unam.mx}
}
\date{16 de Mayo 2020}
\maketitle
%%%%%%%%%%%%%%%%%%%%%%%%%%%%%%%%%%%%%%%%%%%%%%%%%%%%%%%%%%%%%%%%%%%%%%%%%%%%%%%%%%%%%%%%%

%%%%%%%%%%%%%%%%%%%%%%%%%%%%%%%%%%%%%%%%%%%%%%%%%%%%%%%%%%%%%%%%%%%%%%%%%%%%%%%%%%%%%%%%%
\newtheorem{theorem}{Teorema}
\newtheorem{example}{Ejemplo}
\newtheorem{corollary}{Corolario}
\newtheorem{lemma}{Lemma}
\newtheorem{definition}{Definicion}
\newtheorem{prop}{Proposicion}
%%%%%%%%%%%%%%%%%%%%%%%%%%%%%%%%%%%%%%%%%%%%%%%%%%%%%%%%%%%%%%%%%%%%%%%%%%%%%%%%%%%%%%%%%

%%%%%%%%%%%%%%%%%%%%%%%%%%%%%%%%%%%%%%%%%%%%%%%%%%%%%%%%%%%%%%%%%%%%%%%%%%%%%%%%%%%%%%%%%
\begin{enumerate}

\item Demuestre que si $G \leq S_n$ contiene permutaciones impares, entonces $|G|$ es par y exactamente la
mitad de los elementos de $G$ son permutaciones impares.
\begin{proof}
Sea $E$ un subrupo de todas las permutaciones pares de $G$. Sea $\beta$ una permutación impar de $G$
(por hipótesis), entonces cada elemento de la clases lateral izquiera $\beta E$ es una permutación impar
(porque el producto de permutaciones impares con permutaciones pares da una permutación impar). Sea
$\alpha$ una permutación impar de $G$, como $G$ es un subgrupo, entonces hay un elemento $\gamma \in E$ tal
que $\alpha = \beta \gamma$. Como $\alpha$ y $\beta$ son impares, concluimos que $\gamma$ es par y por tanto
$\gamma \in E$.
De ahí que $\alpha \in \beta E$. Por tanto $\beta E$ contiene todas las permutaciones impares de $G$.

Como $|\beta E| = |E|$, se concluye que exactamente la mitad de los elementos son permutaciones pares
y la otra mitad impar, lo cual implica que $G$ es par.
\end{proof}

\item Sea $G$ un grupo finito que contiene un subgrupo $H$ de índice $p$, donde $p$ es el primo más pequeño
que divide al orden de $G$. Demuestre que $H$ es normal en $G$.
\begin{proof}
Sea $H$ el subgrupo de ínidce $p$, donde $p$ es el primo más pequeño que divide al orden de $|G|$. Entonces
$G$ actúa en el conjunto de las clases laterales izquierdas de $H$, $\{ gH \ | \ g \in G \}$ por la
multiplicación izquierda, $x \cdot (gH) = xgH$.

Esa acción induce un homomorfismo $\rho : G \to S_p$, con su kernel que está contenido en $H$\footnote{
    \textbf{Teorema 3.14} \emph{Si $H \leq G$ y $[G:H] = n$, entonces hay un homomorfismo $\rho : G \to S_n$
    con $\ker \rho \leq H$.}
}. Sea $K$ el kernel. Entonces $G/K$ es isomorfa al subgrupo de $S_p$ y entonces tiene un orden que divide
a $p!$. Pero también tiene que tener orden que divida a $G$, y como $p$ es el primo más pequeño que divide
a $|G|$ se sigue que $|G/K| = p$. Como $|G/K| = [G:K] = [G:H][H:K] = p[H:K]$, se sigue que $[H:K] = 1$, 
y de ahí que $K = H$. Como $K$ es normal, $H$ también era normal.
\end{proof}

\begin{definition}
    Un $G$-conjunto es transitivo, si tiene una sola órbita, es decir, para cualesquiera $x, y \in X$, existe
    $g \in G$ tal que $x = gy$.
\end{definition}

\item Si $X$ es un $G$-conjunto, demuestre que cada una de sus órbitas es un $G$-conjunto transitivo.
\begin{proof}
Sea $\mathcal{O}$ una órbita del $G$-conjunto, se tiene por definición que si $g \in G$ y $x \in \mathcal{O}$
entonces $gx \in \mathcal{O}$, y así $\mathcal{O}$ es cerrado bajo producto. Además por definición, para
cualqier $x_1, x_2 \in \mathcal{O}$ existe un $g \in G$ tal que $g x_1 = x_2$, teniendo así que cada órbita
es un $G$-conjunto transitivo.
\end{proof}

\item Demuestre que $A_5$ no tiene subgrupos de orden 30.
\begin{proof}
Supongamos que $A_5$ tiene un subgrupo de orden 30, digamos $H$. Entonces $[A_5 : H] = 2$ lo que implica que
$H$ es normal\footnote{En la tarea 4 se demostró que si \emph{$G$ un grupo y $H$ un subgrupo de $G$ de índice
igual a 2, entonces $H$ es un subgrupo normal de $G$.}}.
Pero como $A_5$ es simple\footnote{\textbf{Teorema 3.11} \emph{$A_n$ es simple para toda $n \geq 5$}.}, lo
cual nos lleva a una contradicción.
\end{proof}
\end{enumerate} 
%%%%%%%%%%%%%%%%%%%%%%%%%%%%%%%%%%%%%%%%%%%%%%%%%%%%%%%%%%%%%%%%%%%%%%%%%%%%%%%%%%%%%%%%%


%%%%%%%%%%%%%%%%%%%%%%%%%%%%%%%%%%%%%%%%%%%%%%%%%%%%%%%%%%%%%%%%%%%%%%%%%%%%%%%%%%%%%%%%%

%%%%%%%%%%%%%%%%%%%%%%%%%%%%%%%%%%%%%%%%%%%%%%%%%%%%%%%%%%%%%%%%%%%%%%%%%%%%%%%%%%%%%%%%%

\end{document} 